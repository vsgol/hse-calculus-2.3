\Subsection{Лекция 1}

\begin{definition}\thmslashn
	
	$f_1 \in H(\Omega_1)$ и $f_2 \in H(\Omega_2)$
	
	$\Delta$ -- компонента связности $\Omega_1 \cap \Omega_2 \not = \emptyset$
	
	Если $f_1\Big|_{\Delta} = f_2\Big|_{\Delta}$, то $f_2$ непосредственное аналитическое продолжение $f_1$ через $\Delta$
	
\end{definition}

\begin{remark}\thmslashn

	При фиксации $\Delta$ продолжение единственно

\end{remark}

\begin{proof}\thmslashn
	
	$\widetilde{f_2} \in H(\Omega_2)\quad \widetilde{f_2}\Big|_{\Delta} = f_1\Big|_{\Delta} = f_2\Big|_{\Delta} \Rightarrow \widetilde{f_2}\Big|_{\Delta} = f_2\Big|_{\Delta} \Rightarrow f_2 \equiv \widetilde{f_2}$ по единственности.
	
\end{proof}


\begin{remark}\thmslashn
	
	Продолжения через разные компоненты связности могут быть разными

\end{remark}


\begin{definition}Продолжение по цепочке областей.
	
	$f\in H(\Omega)$ и $\widetilde{f} \in H(\widetilde{\Omega})$ 
	
	Если существует цеполчка областей
	
	$\Omega_0 = \Omega, \Omega_1, \Omega_2, \ldots, \Omega_n = \widetilde{\Omega}, \quad f_0 = f, \ldots, f_n = \widetilde{f}$
	
	$f_k \in H(\Omega_k)$ и $f_k$ непосредственно аналитическое продолжение $f_{k-1}$
	
\end{definition}

\begin{remarks}\thmslashn
	
	\begin{enumerate}
		\item 
		Результат зависит от выбора компонент связности на каждом шаге
		
		\item
		При их фиксации результат единственный
		
		\item
		Можно считать, что все промежуточные области -- круги
		
	\end{enumerate}
	
\end{remarks}


\begin{definition}\thmslashn
	
	Разобьем множество пар $(f, \Omega)\quad f\in H(\Omega)$
	
	на классы эквивалентности отностительно аналитического продолжения по цепочке. (Очевидно, что это отношение эквивалентности $\Rightarrow$ можем разбить на классы)
	
	Класс эквивалентности -- полная аналитческая функция $F$ 
	
	множество $M := \bigcup_{(f, \Omega)\in F} \Omega$ -- область определения $F$ (существования)
	
\end{definition}

\begin{statement}\thmslashn
	
	$M$ -- область
	
\end{statement}

\TODO написать доказательство
	
\begin{definition}\thmslashn
	
	Значения полной аналитической функции в точке $z\in M$ -- множество значений в точке $z$ всех функций из ее класса эквивалентности.
	
\end{definition}

\begin{remark}\thmslashn
	
	Теорема Пуанкаре-Вольтерры. Множество значений $F$ в точке $z$ либо конечно, либо счетно.
	
	Более того можно ограничиться лишь нбчс $(f, \Omega)$ в классе эквивалентности.
	
\end{remark}

\begin{example}\thmslashn

	$f(z) = \sum\limits_{n=0}^{\infty} z^n$ определена в $\abs{z} < 1$
	
	Можно продолжить до $\frac{1}{1-z}$ в $\CC\setminus \{1\}$

	\TODO Написать про то, как надо дополнять, если мы не знаем формулу.

	$\frac{1}{1-z} = \frac{1}{1-a} \cdot \frac{1}{1-\frac{z-a}{1-a}} = \frac{1}{1-a} \cdot \sum\limits_{n=0}^{\infty}\left( \frac{z-a}{1-a} \right)^n = \sum\limits_{n=0}^{\infty}\frac{(z-a)^n}{(1-a)^{n+1}}$ сходится при $\abs{\frac{z-a}{1-a}} < 1 \Leftrightarrow |z-a| < |1-a|$
	
\end{example}

\begin{definition}\thmslashn
	
	$f(z) = \sum\limits_{n=0}^{\infty} c_n(z-z_0)^n \quad |z-z_0| < R \quad |z_1 - z_0| \leqslant R$
	
	$z_1$ -- правильная точка, если $\exists r > 0$ и $g \in H(B_r(r_1))$, т.ч. $f \equiv g$ в $B_R(z_0) \cap B_r(z_1)$ иначе $z_1$ -- особая точка.
	
\end{definition}

\begin{theorem}\thmslashn 
	
	На границе круга сходимости есть особая точка.
	
\end{theorem}

\begin{proof}\thmslashn
	
	Пусть все точки правильные. 
	
	$S = \{|z-z_0| = R\}$
	
	$\forall z \in S$ найдется круг $B_{r_z}(z)$, на который есть продолжение
	
	У нас покрытие компакта (окружности) $\Rightarrow$ мы можем выбрать конченое подпокрытие шириками, из которых по лемме Лебега возьмем $r>0$ чтобы у любой точки окружности был шарик радиуса $r$
	
	Продолжим $f$ в $\Omega = B_R(z_0) \cap B_{r_1}(z_1)\cap \ldots \cap B_{r_n}(z_n)$
	
	Тогда $f \in H(\Omega)$, но $\Omega \supset B_{R+r}(z_0)$
	
	Разложим $f$ в ряд с центром в $z_0$, он сходится в $B_{R+r}(z_0) \Rightarrow B_R(z_0)$ не круг сходимости.
	
\end{proof}

\begin{example}\thmslashn
	
	\begin{enumerate}
		\item 
		$f(z) = \sum\limits_{n=1}^{\infty} \frac{z^n}{n^2} \quad$ сходится во всех точках $|z| = 1$
		
		$(f(z))' = \sum \frac{z^{n-1}}{n} \quad (zf'(z))' = \sum z^n = \frac{1}{1-z} \Rightarrow$ точка $z = 1$ особая, хотя интеграл сходится
		
		\item
		$\sum\limits_{n=1}^{\infty} z^{2^n} $ все точки $|z| = 1$ особые.
		
	\end{enumerate}
	
\end{example}

\begin{theorem}\thmslashn 
	
	$f\in H(\Omega) \quad \Omega$ -- односвязная область в $\CC$ и $f\not = 0$ в $\Omega$
	
	Тогда существует $g\in H(\Omega)$, т.ч. $e^{g(z)} = f(z) \forall z \in \Omega$ и $g$ единственна с точностью до $+2\pi i k$, где $k \in \Z$
	
\end{theorem}

\begin{proof}\thmslashn
	
	$\frac{f'}{f} \in H(\Omega) \Rightarrow$ у нее есть первообразная в $\Omega$
	
	зафиксируем $z_0 \in \Omega$ и подбурем константу так, что $e^{g(z_0)} = f(z_0)$
	
	Докажем, что $e^{g(z)} = f(z)\,\, \forall z \in \Omega$ 
	
	Рассмотрим $h(z) = e^{-g(z)}f(z)$
	
	$h(z_0) = 1, h'(z) = -g'(z)e^{-g(z)}f(z) + e^{-g(z)}f'(z) = \frac{f'(z)}{f(z)}e^{-g(z)}f(z) + e^{-g(z)}f'(z) = 0$ значит эту функция константа
	
	Пусть есть $g_1$ и  $g_2$, тогда $e^{g_1(z) - g_2(z)} =1 \Rightarrow g_1(z) - g_2(z) = 2\pi i k (z)$ -- непрерывная функция и $k(z)\in \Z \Rightarrow k(z)$ фиксированная функция, тк она не может перескочить с одного дискретного значения на другое (функция непрерывна $\Rightarrow$ теорема Больцана-Коши $\Rightarrow$ есть все значения между, а из нет)
\end{proof}

\begin{consequence}\thmslashn
	
	В односвязной области $\Omega \subset \CC \setminus \{0\}$ существует $g\in H(\Omega)$, т.ч. $e^{g(z)} = z$ и $g$ единственна с точностью до $+2\pi i k $, где $k\in \Z$
	
\end{consequence}

\begin{proof}\thmslashn
	
	Подставим $f(z) = z$
	
\end{proof}

\begin{remark}\thmslashn
	
	$g(z) = \ln |z| + i \arg z$
	
	Причем $\arg z$ непрерывен в $\Omega$
	
\end{remark}

\begin{definition}\thmslashn
	
	Функции $g$ из следствия из одного класса эквивалентности 
	
	Получилась полная аналитическая функция $\Ln$ логарифм
	
	Ее представители -- голоморфные ветви логарифма.
 	
\end{definition}

\begin{properties}[$\text{Ln}$]\thmslashn
	
	\begin{enumerate}
		\item
		$\Ln z = \{w\in \CC : e^w = z\} \quad z \not = 0$
		
		\item
		$\Ln z = \ln |z| + i \Arg z\quad z \not = 0$, где $\Arg = \text{{все аргументы }}z$
		
		\item
		$\Ln (z_1 z_2) = \Ln z_1 + \Ln z_2$
		
		$\Ln (z_1 z_2) = \ln|z_1 z_2| + i \Arg (z_1 z_2) = \ln|z_1| + \ln|z_2| + i \Arg z_1 + i \Arg z_2 = \Ln z_1 + \Ln z_2$
	\end{enumerate}
	
\end{properties}

\begin{remark} Для голоморфногой ветви логарифма свойства 3 нет
	
	$\Ln z = \ln|z| + i \arg z \quad -\pi < \arg z < \pi$
	
	$\Ln e^{3\pi i /4} = \frac{3\pi i}{4}$
	
	$\Ln \left( (e^{3\pi i /4})^2\right) = \Ln \left( e^{2\cdot3\pi i /4}\right) = \Ln \left( e^{3\pi i /2}\right) = -\frac{\pi i}{2} \not = \frac{3\pi i}{4} + \frac{3\pi i}{4}$
	
\end{remark}